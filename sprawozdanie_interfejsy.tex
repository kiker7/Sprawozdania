\documentclass[a4paper,11pt,notitlepage]{article}
\usepackage[utf8]{inputenc}	% latin2 - kodowanie iso-8859-2; cp1250 - kodowanie windows
\usepackage[T1]{fontenc}
\usepackage[polish]{babel}
\usepackage[MeX]{polski}
\usepackage{graphicx}
\selectlanguage{polish}

\usepackage{graphicx}

\hyphenation{FreeBSD}

\author{Rafał Rutyna \\ Piotr Pyśk \\ GR3}
\title{Laboratorium Sieci Komputerowych \\ {\small Interfejsy sieciowe}}
\date{\today}

\linespread{1.3}

\usepackage{indentfirst}

\begin{document}
\maketitle
\newpage
\tableofcontents
\newpage

\section{Zarządzanie interfejsami}
Zadanie wykonywaliśmy na trasowniku połączonym z naszą stacją K5. Aby zalogować się na dany trasownik wpisaliśmy polecenie:
\begin{verbatim}
 k5% ssh t5
 t5%
\end{verbatim}

\subsection{Załadownie sterowników}
Wylistowaliśmy wszyskie urządzenia Ethernet podpięte pod magistralę PCI.

\subsection{Tworzenie i usuwanie interfejsów}
Przed przystąpieniem do pracy sprawdziliśmy spis wszytkich urządzeń podpiętych pod magistralę PCI w celu zweryfikowania 
Stworzyliśmy następujące interfejsy sieciowe: tun, tap, epair i bridge.
Użyliśmy do tego poleceń:
\begin{verbatim}
 t5% # ifconfig tun create
 t5% # ifconfig tap create
 t5% # ifconfig epair create
 t5% # ifconfig bridge create
\end{verbatim}
Po wywołaniu programu \verb+ifconfig -l+ otrzymaliśmy listę interfejsów.
\begin{verbatim}
em0 em1 lo0 tun0 tap0 epair0a epair0b bridge0 
\end{verbatim}
Usunęliśmy je poleceniami
\begin{verbatim}
 t5% # ifconfig tun destroy
 t5% # ifconfig tap destroy
 t5% # ifconfig epair destroy
 t5% # ifconfig bridge destroy
\end{verbatim}

\subsection{Wyłączanie i podnoszenie interfejsów sieciowych}
W celu aktywowania interfejsu wpisaliśmy polecenie:
\begin{verbatim}
 t5% # ifocnfig tun up
\end{verbatim}
Analogicznie aby dezaktywować interfejs należy posłużyć się poleceniem:
\begin{verbatim}
 t5% # ifconfig tun down
\end{verbatim}


\subsection{Ograniczenie przepustowości interfejsu}
Do ograniczenia prędkości transmisji danych wykorzystywanego interfejsu posłużyliśmy się poleceniem:
\begin{verbatim}
 t5% # ifconfig em0 media 10baseTX  
\end{verbatim}
W ten sposób szybkość została ograniczona do 10Mb/s.

\section{Wirtualna sieć komputerowa}
Celem ćwiczenia było stworzenie dwóch maszyn wirtualnych oraz połącznie ich w jedną sieć komputerową.

\subsection{Wykreowanie dwóch maszyn wirtualnych}
Aby stworzyć i uruchomić nową maszynę wirtualną posłużliśmy się gotowym skryptem \verb+vb+ którego użycie wygląda następująco: 
\begin{verbatim}
k5% vb mk kv1
Virtual machine 'kv1' is created and registered.
UUID: d0838d23-73cd-4c17-996b-87a7da33c13c
Settings file: '/home/stud/pyskp/vm/kv1/kv1.vbox'

k5% vb on
Waiting for VM "kv1" to power on...
VM "kv1" has been successfully started.
2090 /usr/local/lib/virtualbox/VBoxHeadless --comment kv1 
--startvm d0838d23-73cd-4c17-996b-87a7da33c13c --vrde config
menu po nacisnieciu F9 ...
\end{verbatim}
Parametr \verb+mk+ odpowiedzialny jest za wykreowanie nowej maszyny, po nim podaję się jej nazwę. Analogicznie stworzyliśmy i uruchomiliśmy drugą. Na pierwszej maszynie podnieśliśmy FreeBSD, a na drugiej GRML. 
Niestety, nie udało się nam stworzyć logów, ponieważ systemy plików były tylko do odczytu.

\subsection{Połączenie maszyn w sieć}
Połączenie dwóch maszyn w sieć 174.16.0.0/24 zrealizowaliśmy poprzez polecenie 
\begin{verbatim}
 % # ifconfig em1 172.16.0.1/24
\end{verbatim}
na maszynie z FreeBSD i polecenie 
\begin{verbatim}
 % ifconfig eth1 172.16.0.2/24
\end{verbatim}
na drugiej. W celu sprawdzenia połączenia uruchomilśmy monitor \verb+tcpdump+ na KV1 na interfejsie em1, a na maszynie KV2 program \verb+ping+. 
Monitor wykrył ruch sieciowy przez obserwowany interfejs, co oznacza połącznie dwóch komputerów.

Następnie położyliśmy na tych samych interfejsach drugą sieć 10.16.0.0/16. Wykonaliśmy to poleceniami
\begin{verbatim}
 % ifconfig em1 alias 10.16.0.1/16
\end{verbatim}
na maszynie z FreeBSD i 
\begin{verbatim}
 % ifconfig eth1:1 inet 10.16.0.2/16
\end{verbatim} i analogicznie sprawdziliśmy istnienie połączenia. Rysunek przedstawia schemat zbudowanej sieci.

\includegraphics[width=1\textwidth]{diagram}

\subsection{Próba uruchomienia WinPE 5.0}
Próbowaliśmy również uruchomić na maszynie KV2 system WinPE 5.0. Pomimo dodawanej wirtualnej pamięci RAM nie udało nam się podnieść tego systemu.

\section{Wnioski}
Wykonując zadane ćwiczenie udało nam się poprawnie skonfigurować interfejsy siecowe oraz wykreować maszyny wirtualne 

\end{document}
